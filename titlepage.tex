%не хотят работать функции для задания шрифта и полей 
%documentclass[a4paper]{article}
%usepackage[14pt]{extsizes} % для того чтобы задать нестандартный 14-ый размер шрифта
%usepackage[utf8]{inputenc}
%usepackage[russian]{babel}
%usepackage{setspace,amsmath}
%usepackage[left=20mm, top=15mm, right=15mm, bottom=15mm, nohead, footskip=10mm]{geometry} % настройки полей документа
\begin{titlepage}
%begin{document}
\begin{center}
\hfill \break
\normalsize{Министерство образования и науки Российской Федерации \\ федеральное государственное бюджетное образовательное учреждение}\\
 
\normalsize{высшего образования}\\
\normalsize{«Иркутский государственный университет» }\\
\normalsize{(ФГБОУ ВО «ИГУ»)}\\
\hfill \break
\normalsize{Институт математики, экономики и информатики}\\
\normalsize{Кафедра Вычислительной математики и оптимизации}\\
\hfill \break
\hfill \break
\hfill \break
\hfill \break
\hfill \break
\hfill \break
\large{\textbf{ВЫПУСКНАЯ КВАЛИФИКАЦИОННАЯ РАБОТА
БАКАЛАВРА}}\\
\hfill \break
\normalsize{
Направление  01.03.02 «Прикладная математика и инфор-матика»\\
профиль Математическое и компьютерное модели-рование }\\
\hfill \break
 
\normalsize{ОПТИМАЛЬНОЕ УПРАВЛЕНИЕ В МОДЕЛЯХ ТРЕЙДИНГА} 
\end{center}
\hfill \break
\hfill \break
\begin{flushright} % сдвигает содержимое окружения вправо
\begin{tabular}{p{.5\textwidth}} % делает таблицу из одной колонки шириной в половину текста


Студентки 4 курса очного отделения\\
группы 2421\\
Дряновой Александры Олеговны\\\

Руководитель:\\
к. в.м. и о., доцент\\
\rule{3cm}{0.25pt}  Сорокин С.П.\\\

Допущена к защите\\
Зав. кафедрой, д. ф.-м. н., профессор\\
 \rule{3cm}{0.25pt}  Дыхта В.А.\\\\

\end{tabular}
\end{flushright}
\hfill \break
\begin{center} Иркутск - 2019 \end{center}
\pagestyle{empty} % % выключаем отображение номера для этой страницы
\end{titlepage}
%end{document}