Прогнозирование, планирование и управление технологическими, социальными, экономическими, биржевыми и другими процессами требует разработки математических методов и моделей для решения задач в различных прикладных областях. Соответственно потребность в разработке новых математических моделях увеличивается, что стало отправной точкой активного исследования и развития нового класса экстремальных задач в теории оптимизации – оптимального управления. По сути это яркий пример того, как запросы практики порождают новые теории и решения.


По мере развития хозяйства уменьшается доля естественных и возрастает промышленных продуктов питания. Пшеница — это зерновая культура, выращиваемая и потребляемая в общемировых масштабах, что делает ее одним из социально значимых мировых продовольственных товаров, охватывающих континенты и поколения. В качестве одного из социально значимых продовольственных товаров, пшеница занимает почти 10\% доли мировых рынков.


Торговля пшеничным зерном не случайно входит в круг интересов не только спекулянтов, но и оптовых продавцов и покупателей, а также её производителей. Именно присутствие спекулянтов, стремящихся играть на разницах котировок, привносит на этот рынок дополнительную ликвидность и является причиной роста волатильности. Уникальность рынка пшеницы в том, что это рынок мирового масштаба, не сосредоточенный в одном конкретном регионе. США являются самым крупным производителем зерна, в то время как Япония самым большим его импортером.
Торговой год в системе реализации зерна играет важную роль. Поэтому трейдерам заранее следует знать, какой операционный торговый год торгуется на текущий момент времени.


Обычно данные по пшенице выпускает Департамент сельского хозяйства США (USDA), чем открывает новый торговый год, длящийся с 01 июня текущего по 30 мая следующего календарного года. Вообще торговые года в системе реализации являются важной составляющей фундаментального анализа рынков зерна (включая пшеницу) и дают аналитикам и трейдерам некоторое представление о сложившемся спросе и предложении.



Трейдеры постоянно отслеживают изменения во внутреннем производстве пшеницы, спросе и ее экспорте, пытаясь понять, как отразятся запасы зерна на фьючерсах на пшеницу к 31 августа.


При приближении окончания текущего торгового года, трейдеры начинают акцентировать внимание на новых данных по производству зерна таких как: Отчет о предполагаемых посевах, где можно просмотреть данные о том, сколько и каких именно сельскохозяйственных культур фермеры планируют посадить в текущем сезоне; Ежемесячные отчеты об урожае; а так же особенно важный Отчет о складских запасах, где сообщается наполненность зернохранилищ.


Таким образом, при выборе более благоприятной стратегии трейдеру необходимо учитывать ряд аспектом торгового года в системе реализации зерна на бирже.

Целью данной дипломной работы является демонстрация возможностей теории оптимального управления в применении к задаче поиска наилучшей стратегии торговли на зерновой бирже.

Задачи исследования:


1.	Рассмотреть и проанализировать базовую модель биржевой торговли зерна. Изучить влияние параметров модели на оптимальную стратегию трейдера. 


2.	Предложить оригинальные модификации исходной модели; дать им экономическую интерпретацию.

3.  Предложены три модификации исходной модели, учитывающие условия хранения зерна. В первой модификации предполагается, что зерно портится (из-за плохих условий хранения) с постоянной скоростью. Во второй модификации скорость порчи зерна является функцией времени и зависит, например, от погодных условий. Третья модификация содержит новое управление, описывающее интенсивность усилий трейдера по повышению условий хранения зерна.

Методы исследования. Для решения поставленных задач в работе использованы методы математического моделирования и оптимального управления.

