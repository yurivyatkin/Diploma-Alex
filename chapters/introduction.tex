На современном этапе развития общества для решения задач в различных областях планирования, прогнозирования, управления технологическими, социальными, экономическими, обменными и другими процессами требуется разработка математических методов и моделей. Потребность в разработке новых математических моделей, активизировала исследования в области теории оптимизации и привело к разработке нового класса экстремальных задач - оптимальное управление. 


По мере развития экономики доля натуральных и промышленных продуктов питания увеличивается. Пшеница - это злак, выращиваемый и потребляемый в глобальном масштабе, что делает его одним из самых социально значимых продуктов питания в мире, охватывающим континенты и поколения. Как один из социально значимых продуктов питания, пшеница занимает почти 10\% мирового рынка.


Торговля зерном пшеницы не случайно входит в круг интересов не только спекулянтов, но и оптовиков и покупателей, а также ее производителей. Именно наличие спекулянтов, стремящихся играть на разнице котировок, приносит дополнительную ликвидность этому рынку и является причиной увеличения волатильности. Уникальность рынка пшеницы заключается в том, что это глобальный рынок, не сконцентрированный в одном конкретном регионе. Соединенные Штаты являются крупнейшим производителем зерна, Россия занимает
третье место в рейтинге мировых экспортеров пшеницы, а Япония - крупнейшим импортером. Торговый год в системе продаж зерна играет важную роль. Поэтому трейдеры должны заранее знать, в каком торговом году идут торги в настоящее время.


Обычно данные по пшенице публикуются Министерством сельского хозяйства США (USDA), которое открывает новый торговый год, который длится с 1 июня текущего по 30 мая следующего календарного года. В целом, торговый год в системе является важной составляющей фундаментального анализа рынков зерна (включая пшеницу) и даёт аналитикам и трейдерам некоторое представление о текущем спросе и предложении.


Трейдеры постоянно следят за изменениями внутреннего производства, спроса и экспорта пшеницы, пытаясь понять, как запасы зерна повлияют на фьючерсы на пшеницу к 31 августа.


Приближаясь к концу текущего торгового года, трейдеры начинают сосредотачиваться на новых данных о производстве зерна, таких как: отчет о перспективных культурах, где вы можете просмотреть данные о том, сколько и каких конкретных культур фермеры планируют посадить в текущем сезоне, ежемесячные отчеты об урожае, а также особенно важный отчет о запасах, в котором сообщается о полноте зернохранилищ.


Таким образом, при выборе более выгодной стратегии трейдер должен учитывать ряд аспектов торгового года в системе продажи зерна на бирже.


Целью данной работы является демонстрация возможностей теории оптимального управления, применительно к задаче нахождения наилучшей торговой стратегии на зерновой бирже.


Цели исследования


1. Рассмотреть и проанализировать базовую модель биржевой торговли зерном. Изучить влияние параметров модели на оптимальную стратегию трейдера.


2. Предложить оригинальные модификации базовой модели; дать им экономическую интерпретацию.


3. Рассмотреть три модификации исходной модели с учетом условий хранения зерна. Первая модификация предполагает, что зерно портится (из-за плохих условий хранения) с постоянной скоростью. Во второй модификации скорость разрушения зерна является функцией времени и зависит, например, от погодных условий. Третья модификация содержит новый контроль, описывающий интенсивность усилий трейдера по улучшению условий хранения зерна.


Для решения поставленных задач были использованы методы математического моделирования и оптимального управления.
