\section{Общее}

Фондовый рынок – сложная динамическая система. Объясняется это тем, что механизмы формирования спроса и предложения и, как следствие, биржевой цены на фондовом рынке определяются коллективной психологией участников рынка, которая, в свою очередь, является ярким примером сложной динамической системы. Именно поэтому, задачи анализа и прогнозирования финансовых временных рядов относятся к классу трудно формулизуемых. Подход к решению такого рода задач заключается в поиске закономерностей.


На сегодняшний день существует огромное количество моделей, методик и инструментов для анализа рынков в целом и отдельных механизмов, функционирующих на нем, в частности. Одним из основных направлений анализа рынков является математическое моделирование. В результате использования экономико-математического моделирования достигается более полное изучение влияния отдельных факторов на обобщающие экономические показатели, уменьшение сроков осуществления анализа, повышается точность осуществления экономических расчетов, решаются многомерные аналитические задачи, которые не могут быть выполнены традиционными методами. 


Математическая модель - это система математических выражений, описывающих поведение реального объекта, составляющих его характеристики и взаимосвязи между ними. Процессом построения математической модели является математическое моделирование. Моделирование и построение математической модели экономического объекта позволяют свести экономический анализ производственных процессов к математическому анализу и принятию эффективных решений. Для этого в планировании и управлении производством необходимо экономическую сущность исследуемого экономического объекта формализовать экономико-математической моделью, т. е. экономическую задачу представить математически в виде уравнений, неравенств и целевой функции на экстремум (максимум и минимум) при выполнении всех условий на ограничения и переменные.


По общему целевому назначению экономико-математические модели строятся для достижения одной из двух целей:


1. Теоретико-аналитические модели предназначаются для изучения общих закономерностей и свойств экономических систем.


2. Прикладные модели строятся для выработки решений конкретных
экономических задач (модели экономического анализа, прогнозирования, управления) При построении используются данные статистических, маркетинговых и других обследований.


По характеру зависимости от времени математические модели делятся на статические модели, характеристики которых не изменяются во времени и динамические – с переменными во времени характеристиками.


Экономические процессы всегда развиваются во времени. Статической экономическая модель получается, если все ее характеристики отнести к одному и тому же моменту времени, динамические модели характеризуют изменения экономических процессов во времени. По длительности рассматриваемого периода различаются модели краткосрочного (до года), среднесрочного (до 5 лет), долгосрочного (10-15 и более лет) прогнозирования и планирования.


Динамические модели в экономике, в свою очередь, делятся на дискретные и непрерывные. В дискретных моделях изменение параметров связано только с отдельными моментами времени. В непрерывных моделях параметры изменяются во времени плавно.


По степени охвата экономические модели подразделяются на макроэкономические и микроэкономические модели. Макроэкономические модели изучают экономику в целом, опираясь на такие укрупненные показатели, как валовой национальный продукт, потребление, инвестиции, занятость, инфляцию, процентную ставку и т.д. Микроэкономические модели описывают экономические процессы на уровне предприятий и фирм, помогая решать стратегические и оперативные вопросы планирования и оптимального управления в рыночных условиях.


По формам представления экономические модели подразделяются на аналоговые, математические и смешанные. Аналоговая модель представляет исследуемый объект аналогом, который ведет себя как реальный объект, но не выглядит как таковой. Пример аналоговой модели - организационная схема. Выстраивая ее, руководство в состоянии представить себе цепи прохождения команд и формальную зависимость между индивидами и деятельностью. Такая аналоговая модель явно более простой и эффективный способ восприятия и проявления сложных взаимосвязей структуры крупной организации, чем, скажем, составление перечня взаимосвязи всех работников. Математическая модель использует символы для описания свойств или характеристик объекта или события. Смешанные модели сочетают в себе свойства аналоговых и математических моделей.


По учету фактора неопределенности модели распадаются на детерминированные, если в них результаты на выходе однозначно определяются управляющими воздействиями, и стохастические (вероятностные), если при задании на входе модели определенной совокупности значений на ее выходе могут получаться различные результаты в зависимости от действия случайного фактора.


По математическому инструменту, применяемому при моделировании, модели классифицируются на методы, применяемые в математических моделях. Наиболее распространенными и эффективными математическими методами, которые нашли как теоретическое, так и практическое приложение в экономических исследованиях, являются: дифференциальное исчисление, математическая статистика, линейная алгебра, математическое программирование, теория графов, теория вероятностей и теория игр.


Порядок построения экономико-математических моделей состоит в следующем: определяется объект исследования (экономика государства в целом, отрасль, предприятие, цех, некоторый социально-экономический процесс, технолого-экономический процесс и т.п.), формулируется цель исследования, определяются характеристики системы, которые должна отображать построенная модель.


 На этапе формализации проводится анализ объекта исследования, определяются его основные структурные и функциональные элементы. Выявляются наиболее существенные характеристики этих элементов, влияющие на достижение поставленной цели моделирования (определяется степень полноты модели). Характеристики системы разделяются на параметры модели (характеристики, которые должны быть известны для построения модели) и переменные модели, которые должны быть определены в результате моделирования.Вводятся символические обозначения используемых величин.Производится математическое описание взаимосвязей между элементами и характеристиками системы – строится собственно экономико-математическая модель.
 
 
На этапе решения в зависимости от цели моделирования и структуры получившейся математической модели выбирается способ проведения расчетов и осуществляется решение задачи.


Различают три вида решения математических моделей:


1. Точное, или аналитическое. При таком решении результат получается в виде готовых формул для вычисления функций или отдельных величин по значениям параметров процесса. Точность полученного решения определяется только точностью вычисления по этим формулам и может, в принципе, быть сколь угодно высокой.


2. Приближенное решение получается с некоторой погрешностью, которая не может быть до конца устранена. Примером приближенного метода решения является графическое решение. Другие приближенные методы могут основываться на упрощении уравнений модели за счет отбрасывания малых слагаемых или разложения функций в ряды по степеням малого параметра с сохранением ограниченного числа членов ряда (особенно часто сохраняется только первый член разложения, так, чтобы задача стала линейной).


3. Численное решение обычно проводится на компьютере. 


Не следует путать точность решения с точностью модели в целом. Точность модели определяется в основном ее полнотой. Даже при наличии точного решения самих уравнений точность модели может оказаться недостаточной.


Проверка адекватности математических моделей обычно осуществляется путем сравнения результатов моделирования с характеристиками реальной системы. Лучше всего для этого попытаться применить модель к какой-то уже существующей системе с известными характеристиками.


Для экономико-математических моделей такой способ проверки редко удается применить на практике (в основном для макроэкономических моделей). Часто модель предполагается адекватной просто на основе того, что в ее основе лежат более или менее достоверные гипотезы (выдвинутые на основе изучения систем и ситуаций, имевших место в прошлом) и более или менее точно определенные параметры. Проверка адекватности такой модели осуществляется постфактум – по результатам последующего функционирования моделируемой системы. Если оказывается, что модель была неадекватна сложившейся ситуации, а на ее основе были приняты конкретные хозяйственные решения – это может быть чревато для системы более или менее значительным кризисом.


Поэтому в экономико-математическом моделировании особенно важен этап постановки задачи. Таким образом, моделирование в экономике является сложной деятельностью, сопряженной с определенными рисками. Тем не менее, в ходе анализа различных экономических систем накоплен значительный опыт построения экономико-математических моделей, доказавших свою адекватность во многих ситуациях.


В современной экономике математика выступает в качестве необходимого инструмента, с помощью которого можно выбрать наилучший вариант действий из многих возможных. Соединение экономики предприятия с математическими расчетами получило название экономико-математических методов . При этом для построения математической модели решения любой экономической задачи существует свой математический метод .


Теория вероятностей: Экономические расчеты, связанные с явлениями и величинами случайного характера .


Математическая статистика: Сбор, обработка и анализ статистических экономических материалов.


Метод статистических испытаний (Монте-Карло): Экономические расчеты, связанные с явлениями и величинами случайного характера, на основе искусственно произведенных статистических материалов.


Теория игр: Выработка экономических решений в условиях неопределенности ситуации, вызванной сознательными злонамеренными действиями конфликтующей стороны.


Теория статистических решений: Выработка экономических решений в условиях неопределенности ситуации, вызванной объективными обстоятельствами.


Задачи оптимального управления – это выбор наиболее выгодных режимов управления сложными динамическими объектами:


•	в механике полета (решение задач оптимизации траекторий полета самолетов и космических кораблей);\\
•	в технике (улучшение технологических процессов, режимов работы роботов);\\
•	в энергетике (оптимизация ядерных реакторов, передача электроэнергии);\\
•	в экономике (оптимальное функционирование макро- и микромоделей экономики);\\
•	в медицине (необходимые программы лечения;\\
•	на основе математических моделей функционирования иммунной, сердечно-сосудистой систем);\\
•	на бирже (поиска стратегий оптимального инвестирования) \\


Теория оптимального управления включает элементы теории управления движением, функционального анализа, исследования операций, математического программирования, теории игр. В широком смысле решить задачу оптимального управления значит разработать для заданного объекта или процесса наилучший закон управления или набор управляющих воздействий. Для этого строится математическая модель объекта или процесса, описывающая его поведение с течением времени под влиянием управляющих воздействий и текущего состояния. Она включает в себя цель управления, выражающуюся через критерий качества; динамику объекта в виде дифференциальных, интегральных, конечно разностных или других уравнений, описывающих способ движения объекта управления; ограничения на используемые ресурсы в виде уравнений или неравенств.


Фундаментальные основы общей теории оптимального управления были заложены в 1956–1961 гг. школами Л.С. Понтрягина [1] и Р. Беллмана [2], хотя, безусловно, на практике задачи оптимального управления встречались и ранее. Центральный результат математической теории оптимального управления – принцип максимума Понтрягина, представляющий собой необходимое условие оптимальности в задаче оптимального управления, был высказан автором в качестве гипотезы в 1955 г., а затем доказан его учениками (Р.В. Гамкрелидзе – для линейного случая, В.Г. Болтянским – для общей нелинейной задачи с функциональными ограничениями). Принцип максимума Понтрягина послужил мощным толчком к пересмотру базовых понятий теории экстремума, к ее развитию и лег в основу огромного количества исследований и новых результатов.


Коллектив американских ученых во главе с Р. Беллманом разработал метод динамического программирования. Он состоит в том, что оптимальное управление строится поэтапно, но при этом на каждом из них будет оптимальным и с точки зрения процесса в целом. Это основное правило динамического программирования, сформулированное Беллманом, и называется принципом оптимальности.


Следует отметить, что практическое применение теории оптимального управления сталкивается с большими трудностями вычислительного характера. Они обусловлены и громоздкостью математических моделей отдельных процессов, и сложностью нахождения аналитических решений. Эти трудности приводят к тому, что построение оптимальных управлений для каждого класса объектов является самостоятельной творческой задачей. Ее решение требует учета специфических особенностей объекта, опыта и интуиции разработчика, в связи с чем мощный толчок получило развитие вычислительных методов в теории оптимального управления. Огромный вклад в их развитие внесли Р.П. Федоренко, Б.Т. Поляк [3], Э. Полак [4]. 


Но несмотря на значительные успехи в этой области, остается много нерешенных задач. Создание новых методов оптимального управления динамическими системами сопряжено с существенными трудностями: неточностями в описании модели, наличием внешних возмущений, помех в устройствах управления и другими факторами. Кроме того, для адекватного моделирования многих современных динамических объектов (в робототехнике, медицине, экономике, народном хозяйстве) требуется использование сложных динамических систем, в частности гибридных систем с разрывной динамикой и систем с распределенными параметрами.


Актуальны исследования, связанные с изучением структурных свойств различных классов динамических систем и разработкой на их основе методов управления, учитывающих наличие неточностей и помех и допускающих коррекцию управляющих воздействий в процессе функционирования системы. Поскольку для осуществления коррекции важно быстро проводить вычисления, учитывающие изменения в динамике объекта, то особенно востребована разработка эффективных вычислительных методов, их построение требует развития новых подходов анализа свойств динамических систем.




\section{Section Title}



\subsection{SubSection Title}


\subsection{SubSection Title}
