\mysubsection{Задача оптимальной торговли зерном.}

В качестве базовой модели выбрана задача, предложенная в Suresh P. Sethi  Optimal control theory: applications to management science and economics , раздел 6.2.2.[8]

Цель поставки данной задачи состоит в максимизации общей стоимости активов трейдора.

Рассмотрим фирму, которая покупает и продает пшеницу. Единственный активы фирмы - денежных средства и пшеница, а цена на пшеницу в течение времени изменяется. 

Цель этой фирмы - покупать и продавать пшеницу, чтобы максимизировать общую стоимость ее активов на горизонте времени Т.


Если фирма хранит слишком мало активов в денежном эквиваленте, то она теряет деньги с точки зрения альтернативных издержек, поскольку она может получить более высокую прибыль, покупая и продавая зерно на бирже.

Для постановки задачи оптимального управления введем следующие обозначения:
 
{Т} = горизонт времени,\\
{х(t)} = остаток денежных средств в 1 денежной единице в момент времени t,\\
{у(t)} = баланс пшеницы в бушелях в момент времени t ( 1 бушель пшеницы = 27,216 кг),\\ 
{v(t)} = скорость покупки пшеницы в бушелях за единицу времени, может быть положительной и отрицательной; отрицательная покупка означает продажу,\\
{р(t)} = цена пшеницы в  1 денежной единице за бушель в момент времени t,\\
{r} = постоянная положительная процентная ставка, заработанная на остатке денежных средств,\\
{h(у)} = стоимость бушеля за время хранения пшеницы.
 
В этом разделе мы не разрешаем {х} и {у} принимать отрицательные значения, так как не допускается заимствования денег и коротких продаж пшеницы.



Запишем уравнения состояния:
\begin{align}
    \Dot{x} & = r x(t) - h(y) - p(t) v(t),\, x(0) = x_{0} \\
    \Dot{y} & = v(t), \, y(0) = y_{0} \\
    - v_2 & \le v(t) \le v_1,
\end{align}    
где {v1} и {v2} неотрицательные постоянные. 

Задача состоит в том, чтобы максимизировать уравнение

\begin{align}
    J = x(T) + p(T) y(T)\to \mathrm{max}
\end{align} 

с учетом (1) - (3). ??? Обратите внимание, что проблема в линейной форме Майера.

\mysubsection{Решение по Принципу максимума Понтрягина}

Составим гамельтониан и лангранжиан. Введем присоединенные переменные $ \psi_1 $  и $ \psi_2 $ :
\begin{align}
    {H} & = \psi_{1} (r x - h(y) - p v) + \psi_{2} (v),\\
    \mathcal{L} & = a_{0} (-x(T) - p(T) y(T)),
\end{align} 
Запишем условия для сопряженных переменных и условия трансверсальности.

\begin{align}
    \Dot{\psi_{1}} & = -H_x = -r \psi_{1},\\
    \Dot{\psi_{2}} & = -H_y = h'(y) \psi_{1},\\
    {\psi_{1}(T)} = - {L}_x(T)=a_{0},\\
    {\psi_{2}(T)} = - {L}_y(T)=a_{0} p(T),
\end{align} 

Интерация решения состоит в следующем:\\
 $ \psi_1(T) $ - это будущая стоимость в момент времени {T} одной денежной единицы, хранящейся на денежном счете с момента времени {t} до {T},\\
 $ \psi_2(T) $ - это будущая стоимость одной денежной единицы, инвестируем в покупку бушеля с момента времени {t} до {T}, с учетом затрат на хранения.\\
 Таким образом, присоединенные переменные имеют естественную интерпретацию, как актуальные оценки конкурентных инвестиций в каждый момент времени.\\ 

Условия максимума функции Понтрягина, выбрав управляющую переменную {v} : \\
\begin{align}
    (\psi_{2} - \psi_{1} p)v\to \mathrm{max}
\end{align} 

Функция переключения:
\begin{align}
g = (\psi_{2} - p \psi_{1})
\end{align} 


"Bang-bang" функция определяется следующим образом:


\begin{align}
v* = 
 \begin{cases}
   v_{1}, & \textrm{ при $g>0$}\\
   -v_{2}, &  \textrm{ при $g<0$}\\
   [-v_{2},v_{1}], & \textrm{ при $g=0$}
 \end{cases}
\end{align}

Поскольку ${v*}$ - это скорость продажи, покупки или бездействия трейдора на бирже при условии оптимальной стратегии, продавать по максимально доступной  ставке стоимости бушеля, или покупать, если стоимость бушеля складывается выгоднее стоимости денежных средств.

В случае если будущая стоимость бушеля с учетом стоимости хранения точно равна биржевой стоимости бушеля, оптимальная стратегия трейдера на бирже не определена.

Таким образом, трейдер покупает, продает или безразличен, если будущая стоимость бушеля, с учетом стоимости хранения меньше, больше или равна биржевой стоимости бушеля.

\mysubsection{Модель продолжительной торговли пшеницей}
Уравнения (1), (2) и (13) определяют двухточечную краевую задачу, которая обычно требует процедуры численного решения. В этом разделе предполагается специальная форма для функции хранения ${h (y)}$, чтобы иметь возможность получить решение в замкнутой форме.

В данном случае мы используем следующие условия:\\
 Стоимость за время хранения пшеницы $$ {h(y)} = \frac{1}{2}|y|$$\\
постоянная положительная процентная ставка, заработанная на денежном остатке $$ {r} = 0, $$\\
остаток денежных средств в одну денежную единицу в момент времени t $$ {x(0)} = 10, $$\\
баланс пшеницы в бушелях в момент времени t $$ {y(0)} = 0, $$\\
скорость покупки пшеницы в бушелях за единицу времени, может быть положительной и отрицательной; отрицательная покупка означает продажу $$ {v_{1}} = {v_{2}} = 1, $$\\
горизонт времени $$ {T} = 6, $$\\
цена пшеницы в одну денежную единицу за бушель в момент времени t

\begin{displaymath}
p(t) =\left\{ \begin{array}{ll}
 3, & \textrm{если $0 \le t\le 3$}\\
 4, & \textrm{если $4  \le t  \le 6$}
  \end{array} \right.
\end{displaymath}


Решение:\\

Запишем уравнения состояния:
\begin{align}
    \Dot{x} & = - \frac{1}{2}|y| - p(t) v(t),\, x(0) = 10 \\
    \Dot{y} & = v(t), \, y(0) = y_{0}\\
    - v_2 & \le v(t) \le v_1,
\end{align}     

Составим гамельтониан. 
\begin{align}
{H} & = \psi_{1} (- \frac{1}{2}|y| - p(t) v(t)) + \psi_{2} (v)
\end{align} 

Запишем условия для сопряженных переменных и условия трансверсальности.
\begin{align}
    \Dot{\psi_{1}} & = -H_x = 0,\\
    \Dot{\psi_{2}} & = -H_y =  \frac{1}{2}\frac{|y|}{y},\\
    {\psi_{1}(6)} = 1,\\
    {\psi_{2}(6)} = 4,
\end{align} 

Условия максимума функции Понтрягина, выбрав управляющую переменную ${v}$ 

\begin{align}
    (\psi_{2} - \psi_{1} p)v\to \mathrm{max}
\end{align} 

Анализ покупки и продажи зерна на предложенном горизонте времени:

\begin{align}
\psi_1=1 \\
\Dot {\psi_{2}}= \frac{1}{2}\frac{|y|}{y} \\
\psi_{2}(6)=4
\end{align} 


Функция переключения:

\begin{align}
g = (\psi_{2} - \psi_{1} {p})
\end{align} 

"Bang-bang" функция определяется следующим образом:

\begin{align}
v* = 
 \begin{cases}
   1, &\textrm{ при $g>0$}\\
   -1, &\textrm{при $ g<0$}\\
   [-1,1], &\textrm{при $g=0$}
 \end{cases}
\end{align}

Таким образом, ${v*} = 1$   скорость покупки бушеля, ${v*} = -1$  скорость продажи бушеля, ${v*  \in (-1, 1)} $ бездействие трейдера на бирже зерна.\\
 
На промежутке времени определим временной период: $ \Delta_{1} = [0,3)$
\begin{align}
\psi_{2}(t) = p(t)\\
\Dot{g} = 0 \rightarrow \Dot{\psi_{2}}(t) = \Dot{p}(t) = 0, \\
\psi_{2} = 3, \\
y = 0, \\
v = 0, 
\end{align}

На промежутке времени определим временной период: $ \Delta_{2} = [3,4]$
\begin{align}
\psi_{2}(t) = p(t)\\
\Dot{g} = 0 , \\
\psi_{2} = 4, \\
y = 0, \\
v = 0, 
\end{align}

При $ на t \in [0,1] $
\begin{align}
\psi_2 = p(t) = 3 
\end{align}

c момента  t* начинаются покупки\\ 
${v} = 0, {y} = 0,$ $\Dot{\psi_2} > 0  \rightarrow g>0 \rightarrow u* = 1 $

\begin{align}
v = 1 \rightarrow y(t) = t - t*, t*>1
\end{align}

\begin{align}
\psi_2(t) =
 \begin{cases}
3  &  \textrm {если $ t \in [0, t*] $}  \\
\frac{1}{2}t + 3 - \frac{1}{2}t*  &  \textrm {если $ t>t* $}
\end{cases}
\end{align}


введем t** - обозначает время конца продаж

\begin{align}
\psi_2 (t**) = \frac{1}{2}t + 3 - \frac{1}{2}t* = 4,\\
t** = 2 + t*,\\
t** - t* = 2, \\
t*>1 \rightarrow t**> 3
\end{align}

\begin{align}
y(t) =
 \begin{cases}
 0 & \textrm{если $t \in [0, t*]$ }\\
 t - t* & \textrm{если $ t > t* $}
  \end{cases}
\end{align}

Предположим, мы начинаем покупать пшеницу при t *> 1. Из (27) коэффициент покупки равен 1; очевидно, что покупка будет продолжаться с такой скоростью.
При достижение  t= 3 покупка бушеля зерна прекращается.
Учитывая затраты на хранения зерна, для минимизации финансовых потерь, зерно должно быть продано в течение 2-х временных единиц после покупки.
После достижения  t = 3  начинается продажа бушелей зерна по максимальной ставке равной 1 и продолжится до момента последней продажи t** (до полной реализации).\\

После $ t=3, v = -1 $ \\

\begin{align}
y(3) = 3 - t*,\\
y(t) = - t + c
\end{align}

Продолжим продажи до момента пока $y(t) = 0$

\begin{align}
y(t**) = 0,\\
-t** + c = 0,\\
c = t**,\\
t* = t \\
\end{align}

Чтобы продать всю закупленную пшеницу необходимо:
\begin{align}
3- t* = t** - 3 
\end{align}


На временном промежутке $ [t**, 6]$\\
 $$v(t) = 0, y(t) = 0, g = 0, \psi_2(t) = 4$$
 
Пока $y(t) > 0 $ на временном промежутке [t*, t**] стоимость хранения зерна будет 
\begin{align}
\psi_2 = \frac{1}{2}. 
\end{align}

Таким образом, v*(t) = 0 в интервале [t**, 6], что также является сингулярным управлением.
Тогда y(t)> 0 для всех  $ t \in (t*, t**)$. Из (3.18), (3.20), $\Dot{\psi_2} = \frac{1}{2}$ в интервале (t*, t**). Чтобы получить единичное управление в интервале (t**, 6), $\psi_2 (t) = 4$ в этом интервале. Кроме того, чтобы иметь сингулярное управление в [0, t*], мы должны иметь  $\psi_2 (t) = 3$ в этом интервале. Теперь мы можем сделать вывод, что из уранения (3.38) $t** - t* = 2$			следует что  t* = 2 и t** = 4. Таким образом, из (3.22)\\


\begin{align}
\overline v(t)=
\left\{ \begin{array}{ll}
 0 & \textrm{при $ t \in [0, 2]$}\\
 1 & \textrm{при $ t \in  (2, 3]$}\\
  -1 & \textrm{при $ t \in  (3, 4]$}\\
   0 & \textrm{при $ t \in  (4, 6]$}
  \end{array} \right.
\end{align}

\begin{align}
 \overline y(t)=\left\{ \begin{array}{ll}
 0 & \textrm{при $ t \in [0, 2]$}\\
 t - 2  & \textrm{при $ t \in  (2, 3]$}\\
  -t + 4 & \textrm{при $ t \in  (3, 4]$}\\
   0 & \textrm{при $ t \in  (4, 6]$}
  \end{array} \right.
\end{align}

\begin{align}
 \overline x(t)=\left\{ \begin{array}{ll}
 10 & \textrm{при $ t \in [0, 2]$}\\
 \frac{27}{4} & \textrm{при $ t \in  (2, 3]$}\\
  10,5 & \textrm{при $ t \in  (3, 4]$}\\
   10,5 & \textrm{при $ t \in  (4, 6]$}
  \end{array} \right.
\end{align}

Для достижения основной цели - максимизации общей стоимости активов, на предложенном горизонте времени ${T = 6}$ определены временные периоды покупки [2, 3] оптимально выгодной; период оптимально выгодной продажи [3,4].

Установлено, что оптимальная стратегия позволяет увеличить общую стоимость активов с 10 до 10,5 денежных единиц в конечный момент времени по сравнению со стратегией бездействия трейдера. Причина этого состоит в том, что прогнозируемый скачок цены зерна дает возможность получить дополнительную выгоду при заданном уровне затрат на хранение пшеницы.
