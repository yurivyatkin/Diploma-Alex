\mysubsection{Математическое моделирование }

Современная математика характеризуется интенсивным использованием ее в различных науках. Математика стала для многих отраслей знаний не только инструментом количественного расчета, но также и методом точного исследования и средством формулировки задач исследования.


В настоящее время в различных областях естествознания математическое моделирование становится основным способом исследования и получения новых знаний. Сегодня цели и задачи математического моделирования очень широки и многообразны, но могут быть кратко сформулированы как качественное и количественное изучение всего, что нас окружает, всевозможных объектов природы, техники и общества. В дальнейшем следует ожидать, что и в других областях человеческой деятельности роль математического моделирования будет возрастать. В связи с этим возникает вопрос о рациональном использовании возможностей математического моделирования.


Математическая модель - это совокупность математических объектов и соотношений между ними, адекватно отображающая свойства и поведение исследуемого объекта. Математическая модель воспроизводит необходимым образом выбранные стороны физической ситуации, в том случаи если можно установить правило соответствия, связывающее характерные особенности физических объектов и отношения с определенными математическими объектами и отношениями.


Математическое моделирование–это идеальное научное знаковое формальное моделирование, при котором описание объекта осуществляется на языке математики, а исследование модели проводится с использованием тех или иных математических методов. 


Математическое моделирование имеет следующие преимущества:


Экономичность (в частности, сбережение ресурсов реальной системы


Возможность моделирования гипотетических, то есть нереализуемых в природе объектов (прежде всего на разных этапах проектирования);


Возможность реализации режимов опасных или трудно воспроизводимых в натуре (критический режим ядерного реактора, работа системы противоракетной обороны);


Возможность изменения масштабов времени; простота много аспектного анализа;


Большая прогностическая сила в следствие возможности выявления общих закономерностей;


Универсальность технического и программного обеспечения проводимой работы (ЭВМ, системы программирования и пакеты прикладных программ широкого назначения);


При построении математической модели, изучаемого объекта или явления выделяют те его особенности, черты и детали, которые с одной стороны содержат более или менее полную информацию об объекте, а с другой допускают математическую формализацию. Математическая формализация означает, что особенностям и деталям объекта можно поставить в соответствие подходящие адекватные математические понятия: числа, функции, матрицы и так далее. Тогда связи и отношения, обнаруженные и предполагаемые в изучаемом объекте между отдельными его деталями и составными частями можно записать с помощью математических отношений: равенств, неравенств, уравнений. В результате получается математическое описание изучаемого процесса или явление, то есть его математическая модель.


Основные этапы моделирования


1. Постановка цели, сбор данных об объекте исследований


Определяются цели и задачи исследования, согласно которым выбирается объект исследования — носитель свойств и качеств, подлежащих изучению 


 Подбирается или разрабатывается соответствующая теория. Устанавливаются причинно-следственные связи между переменными описывающими объект. Определяются входные и выходные данные, принимаются упрощающие предположения. На первом этапе математического моделирования такую расчетную схему понимают как условное описание объекта исследований, которое должно учитывать его особенности и количественные характеристики, существенные для рассматриваемого случая. При этом обосновывают упрощения, которые позволяют не учитывать несущественные свойства и качества объект исследования.
Полнота и правильность учета существенных свойств и качеств объекта исследования являются одним из основных условий, получения в дальнейшем достоверных результатов. Успешное проведение этого неформального этапа в значительной мере зависит от опыта и профессиональных качеств исследователя.


2. Формализация.


. На втором этапе осуществляют математическое описание расчетной схемы. Выбираются системы условных обозначений и с их помощью записывать отношения между составляющими объекта в виде математических выражений. Такое формальное описание называют математической моделью. Устанавливается класс задач, к которым может быть отнесена полученная математическая модель объекта. Значения некоторых параметров на этом этапе еще могут быть не конкретизированы.


Объект исследования может иметь несколько математических моделей. Это связано, прежде всего с тем, что рассматриваемые  расчетные схемы, отличаются уровнем упрощения. Однако даже при рассмотрении одной и той же расчетной схемы можно построить принципиально разные математические модели.


В тоже время, одна и та же математическая модель может соответствовать расчетным схемам, описывающим объекты из различных предметных областей, что облегчает изучение такой модели. Кроме того, для многих типовых расчетных схем уже построены математические модели, что упрощает проведение второго этапа.


В большинстве случаев желательно разработать вариант математической моделью, который позволяет получить точное или привлечь известное решение. В дальнейшем такое решение используют при реализации некоторых этапов математического моделирования.


3. Выбор рабочей математической модели.


На третьем этапе математического моделирования проводят качественный и оценочный количественный анализ построенной математической модели. При этом могут быть выявлены противоречия, устранение которых потребует уточнения или пересмотра расчетной схемы. Итог анализа на рассматриваемом этапе — выбор рабочей математической модели.


4. Выбор метода решения.


На этом этапе осуществляют обоснованный выбор или построение численного метода. Для решения одной и той же вычислительной задачи обычно может быть использовано несколько методов.  При выборе метода учитываются требования, предъявляемые к решению,  наличии ресурсов, знания пользователя, его предпочтения, предпочтения разработчика и др.


5. Реализация модели.


Разрабатывается эффективный вычислительный алгоритм, при помощи, которого результат может быть получен с необходимой точностью и за доступное время.


К настоящему времени разработан ряд требований к вычислительным алгоритмам: корректность и хорошая, надлежащая точность, экономичность, простота и др.


Разрабатывают программу, реализующую вычислительный алгоритм. К программам также предъявляют ряд требований: надежность, работоспособность, переносимость, простота в использовании и др.
После устранения всех выявленных недочетов приступают к проведению вычислительного эксперимента.


6. Анализ полученной информации. 


Сопоставляется полученное и предполагаемое решение, проводится контроль погрешности моделирования


7. Проверка адекватности реальному объекту.


Результаты, полученные по модели сопоставляются либо с имеющейся об объекте информацией или проводится эксперимент и его результаты сопоставляются с расчётными.


\mysubsection{Основные принципы построения математических моделей}


Для более эффективного использования математической модели, она должна обладать определенными свойствами:


1. Полнота математической модели. Полнота математической модели позволяет в достаточной мере отразить существенные в данном случае свойства и качества объекта исследований. 


2. Точность математической модели.  


3. Адекватности. Под этим свойством математической модели понимают правильное качественное и достаточно точное количественное описание представляющих интерес характеристик объекта исследований. В технике адекватность понимают, как способность математической модели описывать количественные характеристики объекта исследований, которые представляют интерес при проведении исследования, с погрешностью не более заданного значения.
В ряде областей человеческой деятельности под адекватностью математической модели могут понимать только правильное качественное описание представляющих интерес характеристик объекта исследований.


4. Свойство продуктивности. Продуктивность математической модели связывают с возможностью располагать достаточно достоверными исходными данными. Если такой возможности нет, то модель будет непродуктивной и ее применение утрачивает всякий смысл.


5. Свойство экономичности. Экономичность математической модели определяют необходимыми затратами времени и ресурсов для изучения математической модели. Например, это свойство означает, что модель не требует при проведении вычислительного эксперимента больших затрат машинного времени и памяти. Чем более простой является математической модели, тем в большей степени она обладает этим свойством.


6. Свойство робастности. Это свойство характеризует способность не допускать чрезмерного влияния погрешности исходных данных на результаты исследования.
Для создания математических моделей, удовлетворяющих всем приведённым выше свойствам, необходимо воспользоваться следующими основными принципами построения математической модели.


1. Построения математической модели с достаточно ограниченной областью адекватности. В начале исследования обычно не известен необходимый диапазон изменения параметров объекта исследований. Чем шире диапазон изменения параметров объекта исследований, тем обычно сложнее становится математической модели. Например, если математической модели рассматриваемого объекта исследований является линейной, то расширение диапазона изменения параметров объекта исследований может стать причиной возникновения нелинейности, что приведет к замене линейной математической модели более сложной — нелинейной. Построение и дальнейшее изучение модели с более широкой областью адекватности требует дополнительных затрат времени и средств, причем может оказаться, что в данном исследовании такая математической модели не нужна. 


2. Принцип постепенного усложнения математической модели. 


Согласно данному принципу построение модели рассматриваемого объекта исследований необходимо начинать с простейших математической модели и проверки их пригодности. Если модель признают пригодной, то ее используют на последующих этапах математического моделирования. Если она не пригодна, то необходимо осуществить следующий цикл модификации модели, который приведет к построению более сложной математической модели и проверке ее пригодности, и так далее до тех пор, пока не будет получена пригодная математической модели. 


3. Принцип согласованности. 

Точность математической модели должна быть согласована с погрешностью исходных данных. Это означает (при прочих равных условиях), что чем выше погрешность исходных данных, тем менее точной должна быть математической модели. Чем меньше величина погрешности исходных данных, тем более точной может быть математическая модель.
Если погрешность исходных данных выше значения количественной характеристики точности модели, то модель не будет отвечать требованию продуктивности и ее использование теряет всякий смысл. Если погрешность исходных данных намного меньше значения количественной характеристики точности модели, то модель может быть недостаточно точной и не отвечать требованию полноты.
Основываясь на принципе согласованности и применяя следующий принцип, можно обосновать необходимость получения более точных исходных данных или аргументировано отказаться от их использования.


4. Принцип перехода к стохастической математической модели.

Некоторые величины или зависимости, используемые при построении модели, иногда невозможно или затруднительно установить с требуемой точностью. Тогда возникает потребность рассмотрения случайных величин или случайных функций, что приводит к появлению стохастической математической модели.
Основная трудность при анализе стохастической математической модели обычно связана с тем, что необходимые характеристики случайных величин или случайных функций часто не известны или известны с невысокой точностью. Это означает, что модель не удовлетворяет требованию продуктивности. Тогда в соответствии с принципом согласованности следует использовать модель, менее точную по сравнению со стохастической, но зато в большей мере удовлетворяющую свойству робастности.


\mysubsection{ Классификация математических моделей}


Стремительное развитие методов математического моделирования и многообразие областей их применения привели появлению большого количества моделей различных видов и к необходимости классификации моделей по тем категориям, которые являются универсальными для всех моделей или необходимы в той области знаний для применения в которой разработаны модели, например: наличие или отсутствие случайных (или неопределенных) факторов; вид критерия эффективности и наложенных ограничений; отрасль знаний; способ представления моделей; и т.д.


Приведем пример классификации математических моделей по некоторым категориям


1. По фактору времени и области использования, выделяют статические и динамические модели. Если все входящие в модель величины не зависят от времени, то имеем статическую модель объекта или процесса (одномоментный срез информации по объекту). Динамическая модель позволяет увидеть изменения объекта во времени.


2.По характеру модулируемого процесса, выделяют  детерминированные и стохастические математические модели. В детерминированных моделях все взаимосвязи, переменные и константы заданы точно, что приводит к однозначному определению результирующей функции. Детерминированная модель строится в тех случаях, когда факторы, влияющие на исход операции, поддаются достаточно точному измерению или оценке, а случайные факторы либо отсутствуют, либо ими можно пренебречь.
Если часть или все параметры, входящие в модель по своей природе являются случайными величинами или случайными функциями, то модель относят к классу стохастических моделей. В стохастических моделях задаются законы распределения случайных величин, что приводит к вероятностной оценке результирующей функции и реальность отображается как некоторый случайный процесс, ход и исход которого описывается теми или иными характеристиками случайных величин: математическими ожиданиями, дисперсиями, функциями распределения и т.д. Построение такой модели возможно, если имеется достаточный фактический материал для оценки необходимых вероятностных распределений или если теория рассматриваемого явления позволяет определить эти распределения теоретически (на основе формул теории вероятностей, предельных теорем и т.д.).


3. По целям моделирования различают дескриптивные, оптимизационные и управленческие модели. В дескриптивных (от лат. descriptio – описание) моделях исследуются законы изменения параметров модели. 
Оптимизационные модели применяются для определения наилучших (оптимальных), на основе некоторого критерия, параметров моделируемого объекта или способов управления этим объектом. Оптимизационные модели строятся с помощью одной или нескольких дескриптивных моделей и имеют несколько критериев определения оптимальности. На область значений входных параметров могут быть наложены ограничения в виде равенств или неравенств, связанных с особенностями рассматриваемого объекта или процесса. Примером оптимизационной модели служит составление рациона питания в определенной диете 
Управленческие модели применяются для принятия решений в различных областях деятельности человека, когда из всего множества альтернатив выбирают несколько и общий процесс принятия решения представляет собой последовательность таких альтернатив. Так как оптимальность принятого решения в одной и той же ситуации может трактоваться различным образом, то вид критерия оптимальности в управленческих моделях заранее не фиксируется. Методы формирования критериев оптимальности в зависимости от вида неопределенности рассматриваются в теории выбора и принятия решений, базирующейся на теории игр и исследовании операций.


4. В зависимости от числа сторон, принимающих решение, выделяют два типа математических моделей: описательные и нормативные. В описательной модели нет сторон, принимающих решения. Формально число таких сторон в описательной модели равно нулю. Типичным примером подобных моделей является модели систем массового обслуживания. Для построения описательных моделей может также использоваться теория надежности, теория графов, теория вероятностей, метод статистических испытаний (метод Монте-Карло).


Для нормативной модели характерно множество сторон. Принципиально можно выделить два вида нормативных моделей: модели оптимизации и теоретико-игровые. В моделях оптимизации основная задача выработки решений технически сводится к строгой максимизации или минимизации критерия эффективности, т.е. определяются такие значения управляемых переменных, при которых критерий эффективности достигает экстремального значения (максимума или минимума).


Для выработки решений, отображаемых моделями оптимизации, наряду с классическими и новыми вариационными методами (поиск экстремума) наиболее широко используются методы математического программирования (линейное, нелинейное, динамическое). Для теоретико-игровой модели характерна множественность числа сторон (не менее двух). Если имеются две стороны с противоположными интересами, то используется теория игр, если число сторон более двух и между ними невозможны коалиции и компромиссы, то применяется теория бескоалиционных игр n лиц.


Классификация моделей по какому-либо одному признаку не может охватить всех видов моделей, ибо модель, как и исходная система, многогранна и отражает лишь те ее свойства, которые представляют интерес для исследователя.


\mysubsection{ Оптимальное управление}


В XX веке при огромном размахе производства и осознании ограниченности ресурсов Земли во весь рост встала задача оптимального использования энергии, материалов, рабочего времени, большую актуальность приобрели вопросы наилучшего в том или ином смысле управления различными процессами физики, техники, экономики и др.


 Развитие теории оптимального управления связано с ростом требований как к быстродействию и точности систем регулирования, так и переходом к рыночной экономике. Увеличение быстродействия возможно лишь при правильном распреде лении ограниченных ресурсов управления, и поэтому учет ограничений на управление стал одним из центральных в теории оптимального управления. С другой стороны, построение систем регулирования высокой точности привело к необходимости учета при синтезе регуляторов взаимовлияния отдельных частей (каналов) системы. Синтез таких сложных многомерных (многосвязных) систем также составляет предмет теории оптимального управления.
 
 
Задачи оптимального управления – это выбор наиболее выгодных режимов управления сложными динамическими объектами: в механике полета; в технике; в энергетике; в экономике; на бирже (поиска стратегий оптимального инвестирования) и тд.


Теория оптимального управления включает элементы теории управления движением, функционального анализа, исследования операций, математического программирования, теории игр. В широком смысле решить задачу оптимального управления значит разработать для заданного объекта или процесса наилучший закон управления или набор управляющих воздействий. Для этого строится математическая модель объекта или процесса, описывающая его поведение с течением времени под влиянием управляющих воздействий и текущего состояния. Она включает в себя цель управления, выражающуюся через критерий качества; динамику объекта в виде дифференциальных, интегральных, конечноразностных или других уравнений, описывающих способ движения объекта управления; ограничения на используемые ресурсы в виде уравнений или неравенств.


Фундаментальные основы общей теории оптимального управления были заложены в 1956–1961 гг. школами Л.С. Понтрягина  и Р. Беллмана, хотя, безусловно, на практике задачи оптимального управления встречались и ранее. Центральный результат математической теории оптимального управления – принцип максимума Понтрягина, представляющий собой необходимое условие оптимальности в задаче оптимального управления, был высказан автором в качестве гипотезы в 1955 г., а затем доказан его учениками (Р.В. Гамкрелидзе – для линейного случая, В.Г. Болтянским – для общей нелинейной задачи с функциональными ограничениями). Принцип максимума Понтрягина послужил мощным толчком к пересмотру базовых понятий теории экстремума, к ее развитию и лег в основу огромного количества исследований и новых результатов.


Значение математической теории оптимальных процессов управления заключается в том, что она дает единую методологию решения весьма широкого круга задач оптимального проектирования и управления, устраняет инерции и недостаточную общность прежних частных методов и способствует ценным результатам и методам, полученным в смежных областях.
