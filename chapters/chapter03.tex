\section{Задача оптимальной торговли зерном.}

Рассмотрим фирму, которая покупает и продает пшеницу. Единственный активы фирмы - денежных средства и пшеница, а цена на пшеницу в течение времени изменяется. Цель этой фирмы - покупать и продавать пшеницу, чтобы максимизировать общую стоимость ее активов на горизонте времени Т.  

Введем следующие обозначения:
 
{Т} = горизонт времени,\\
{х(t)} = остаток денежных средств в долларах в момент времени t,\\
{у(t)} = баланс пшеницы в бушелях в момент времени t,\\ 
{u(t)} = скорость покупки пшеницы в бушелях за единицу времени; отрицательная покупка означает продажу,\\
{р(t)} = цена пшеницы в долларах за бушель в момент времени t,\\
{r} = постоянная положительная процентная ставка, заработанная на денежном балансе,\\
{h(у)} = стоимость за время хранения пшеницы.
 
В этом разделе мы разрешаем {х} и {у} принимать отрицательные значения, допускаются значения денег и короткие продажи пшеницы.



Запишем уравнения состояния:
\begin{align}
    \Dot{x} & = r x(t) - h(y) - p(t) v(t),\, x(0) = x_{0} \\
    \Dot{y} & = v(t), \, y(0) = y_{0} \\
    - v_2 & \le v(t) \le v_1,
\end{align}    
где {v1} и {v2} неотрицательные постоянные. Задача состоит в том, чтобы максимизировать уровнение
% вообще-то, это не по-русски, и надо переписать по-другому:
\[
  \mathrm{maximize} [J = x(T) + p(T) y(T)]
\]
с учетом (3.1) - (3.3). Обратите внимание, что проблема в линейной форме Майера.
% Не все знают, что такое линейная форма Майера, и почему это хорошо. Об этом стоит написать в главе про основные понятия и теоремы.

\subsection{Решение по Принципу максимума Понтрягина}

Составим гамельтониан и лангранжиан. 
\begin{align}
    \mathcal{H} & = \psi_{1} (r x - h(y) - p v) + \psi_{2} (v),\\
    \mathcal{L} & = a_{0} (-x(T) - p(T) y(T)),\\
\end{align} 
Запишем условия для сопряженных переменных и условия трансверсальности.
\begin{align}
    \Dot{\psi_{1}} & = -H x = -r \psi_{1},\\
    \Dot{\psi_{2}} & = -H y = h'(y) \psi_{1},\\
    {\psi_{1}(T)} = - {L}x(T)=a_{0},\\
    {\psi_{2}(T)} = - {L}y(T)=a_{0} p(T),
\end{align} 

Условия максимума функции Понтрягина.
\begin{align}
    (\psi_{2} - \psi_{1} p)v\to \mathrm{max}
\end{align} 

Функция переключения:
$g = (\psi_{2} - p)$\\

"Bang-bang" функция определяется следующим образом:

\begin{align}
v* = 
 \begin{cases}
   v_{1}, &\text{g>0}\\
   -v_{2}, &\text{g<0}\\
   [-v_{2},v_{1}], &\text{g=0}
 \end{cases}
\end{align}


???

\subsection{Модель продолжительной торговли пшеницей}
Уравнения (3.1), (3.2) и (3.12) определяют двухточечную краевую задачу, которая обычно требует процедуры численного решения. В этом разделе мы предполагаем специальную форму для функции хранения h (y), чтобы иметь возможность получить решение в замкнутой форме.

В данном случае мы используем следующие условия:\\
 Стоимость за время хранения пшеницы $$ h(y) = \frac{1}{2}|y|$$\\
постоянная положительная процентная ставка, заработанная на денежном балансе $$ r = 0, $$\\
остаток денежных средств в долларах в момент времени t $$ x(0) = 10, $$\\
баланс пшеницы в бушелях в момент времени t $$ y(0) = 0, $$\\
скорость покупки пшеницы в бушелях за единицу времени; отрицательная покупка означает продажу $$ V_{1} = V_{2} = l, $$\\
горизонт времени $$ T = 6, $$\\
цена пшеницы в долларах за бушель в момент времени t
\begin{displaymath}
p(t) =\left\{ \begin{array}{ll}
 3 & \textrm{если $0 \le t\le 3$}\\
 4 & \textrm{если $4  \le t  \le 6$}
  \end{array} \right.
\end{displaymath}


решение:\\

Запишем уравнения состояния:
\begin{align}
    \Dot{x} & = - \frac{1}{2}|y| - p(t) v(t),\, x(0) = 10 \\
    \Dot{y} & = v(t), \, y(0) = y_{0}\\
    - v_2 & \le v(t) \le v_1,
\end{align}     

Составим гамельтониан. 
\begin{align}
    \mathcal{H} & = \psi_{1} (- \frac{1}{2}|y| - p(t) v(t)) + \psi_{2} (v)
\end{align} 

Запишем условия для сопряженных переменных и условия трансверсальности.
\begin{align}
    \Dot{\psi_{1}} & = -H x = 0,\\
    \Dot{\psi_{2}} & = -H y =  \frac{1}{2}\frac{|y|}{y},\\
    {\psi_{1}(6)} = 1,\\
    {\psi_{2}(6)} = 4,
\end{align} 

Условия максимума функции Понтрягина.
\begin{align}
    (\psi_{2} - \psi_{1} p)v\to \mathrm{max}
\end{align} 

Анализ:
$$\psi_1=1$$
$$\Dot {\psi_{2}}= \frac{1}{2}\frac{|y|}{y}$$
$$\psi_{2}(6)=4$$ 
Функция переключения:\\
$$ g = (\psi_{2} - p) $$\\

"Bang-bang" функция определяется следующим образом:

\begin{align}
v* = 
 \begin{cases}
   1, &\text{g>0}\\
   -1, &\text{g<0}\\
   [-1,1], &\text{g=0}
 \end{cases}
\end{align}

 
на промежутке: $$ \Delta_{1} = [0,3)$$
\begin{align}
\psi_{2}(t) = p(t)\\
\Dot{g} = 0 \rightarrow \Dot{\psi_{2}}(t) = \Dot{p}(t) = 0, \\
\psi_{2} = 3, \\
y = 0, \\
v = 0, \\
\end{align}

на промежутке: $$ \Delta_{2} = [3,4]$$
\begin{align}
\psi_{2}(t) = p(t)\\
\Dot{g} = 0 , \\
\psi_{2} = 4, \\
y = 0, \\
v = 0, \\
\end{align}

$$ на t \in [0,1],$$
$$\psi_2 = p(t) = 3 $$\\
c момента  t* начинаются покупки\\ 
v = 0, y = 0, $$\Dot{\psi_2} > 0  \rightarrow g>0 \rightarrow u* = 1$$

\begin{align}
v = 1 \rightarrow y(t) = t - t*, t*>1
\end{align}

\begin{displaymath}
\psi_2(t) = \left\{ \begin{array}{ll}
3  &  \textrm {если $ t \in [0, t*] $}  \\
\frac{1}{2}t + 3 - \frac{1}{2}t*  &  \textrm {если $ t>t* $}
\end{array} \right.
\end{displaymath}

введем t** - обозначает время конца продаж

\begin{align}
\psi_2 (t**) = \frac{1}{2}t + 3 - \frac{1}{2}t* = 4,\\
t** = 2 + t*,\\
t** - t* = 2, \\
t*>1 \rightarrow t**> 3
\end{align}

\begin{displaymath}
y(t) =\left\{ \begin{array}{ll}
 0 & \textrm{если $t \in [0, t*]$ }\\
 t - t* & \textrm{если $ t > t* $}
  \end{array} \right.
\end{displaymath}

Предположим, мы начинаем покупать пшеницу при t *> 1. Из (3.22) коэффициент покупки равен 1; очевидно, что покупка будет продолжаться с такой скоростью, пока t= 3, и больше не будет. Чтобы не потерять деньги из-за хранения пшеницы, она должна быть продана в течение 2-х временных единиц после покупки. Ясно, что мы должны начать продавать с t = 3 по максимальной ставке 1 и продолжить до момента последней продажи t**.\\

После $ t=3, v = -1 $ \\

\begin{align}
y(3) = 3 - t*,\\
y(t) = - t + c
\end{align}

Продолжим продажи до момента пока $y(t) = 0$

\begin{align}
y(t**) = 0,\\
-t** + c = 0,\\
c = t**,\\
t* = t \\
3- t* = t** - 3 
\end{align}


 на временном промежутке $ [t**, 6]$\\
 $$v(t) = 0, y(t) = 0, g = 0, \psi_2(t) = 4$$
пока $y(t) > 0 $ стоимость хранения зерна будет $$\psi_2 = \frac{1}{2}$$ на временном промежутке [t*, t**]

Таким образом, v*(t) = 0 в интервале [t**, 6], что также является сингулярным управлением.
Тогда y(t)> 0 для всех  $ t \in (t*, t**)$. Из (3.18), (3.20), $\Dot{\psi_2} = \frac{1}{2}$ в интервале (t*, t**). Чтобы получить единичное управление в интервале (t**, 6), $\psi_2 (t) = 4$ в этом интервале. Кроме того, чтобы иметь сингулярное управление в [0, t*], мы должны иметь  $\psi_2 (t) = 3$ в этом интервале. Теперь мы можем сделать вывод, что из уранения (3.38) $t** - t* = 2$			следует что  t* = 2 и t** = 4. Таким образом, из (3.22)\\


\begin{align}
\overline v(t)=
\left\{ \begin{array}{ll}
 0 & \textrm{при $ t \in [0, 2]$}\\
 1 & \textrm{при $ t \in  (2, 3]$}\\
  -1 & \textrm{при $ t \in  (3, 4]$}\\
   0 & \textrm{при $ t \in  (4, 6]$}
  \end{array} \right.
\end{align}

\begin{align}
 \overline y(t)=\left\{ \begin{array}{ll}
 0 & \textrm{при $ t \in [0, 2]$}\\
 t - 2  & \textrm{при $ t \in  (2, 3]$}\\
  -t + 4 & \textrm{при $ t \in  (3, 4]$}\\
   0 & \textrm{при $ t \in  (4, 6]$}
  \end{array} \right.
\end{align}

\begin{align}
 \overline x(t)=\left\{ \begin{array}{ll}
 10 & \textrm{при $ t \in [0, 2]$}\\
 \frac{27}{4} & \textrm{при $ t \in  (2, 3]$}\\
  10,5 & \textrm{при $ t \in  (3, 4]$}\\
   10,5 & \textrm{при $ t \in  (4, 6]$}
  \end{array} \right.
\end{align}
\subsection{SubSection Title}

