В данной главе предложены три модификации, учитывающие условия хранения зерна. В первой модификации предполагается, что зерно портится (из-за плохих условий хранений) с постоянной скоростью. Во второй модификации скорость порчи зерна является функцией времени и зависит, например, от погодных условий. Третья модификация содержит новое управление, описывающее интенсивность усилий трейдера по повышению условий хранения зерна.

Дальнейшая работа будет связана с анализом предложенных модификаций.


Обозначения:\\
{\begin{math} T \end{math} --- горизонт времени,\\
\emph {х}(\emph{t}) --- остаток денежных средств в момент времени $t$ (фазовая переменная),\\
\emph {у}(\emph{t})  --- баланс пшеницы в бушелях в момент времени $t$ ( 1 бушель пшеницы = 27,216 кг)(фазовая переменная),\\ 
${v(t)}$ --- скорость покупки пшеницы в бушелях за единицу времени, может быть положительной и отрицательной; отрицательная покупка означает продажу (управление),\\
\emph {р}(\emph{t}) --- цена пшеницы за бушель в момент времени $t$,\\
${r}$ --- постоянная положительная процентная ставка, на остаток денежных средств на счете,\\
\emph {h}(\emph{y}) --- функция затрат на хранение пшеницы.
 
Фазовые переменные $x$ и $y$ могут принимать отрицательные значения, таким образом допускаются заимствование денег и короткие продажи пшеницы.


\mysubsection{Первая модификация задачи}
Добавим в исходную задачу коэффициент влияния условий хранения как постоянное число.\\

Введем новые обозначение:\\
${a}$ = коэффициент влияния условий хранения на зерно.\\
$a$ - const\\
Наша задача запишется так:

\begin{align}
\mathcal{J} & = x(T) + p(T) y(T) \to \mathrm{max},\\
    \Dot{y} & = v(t) - a y, & y(t_{0}) = y_{0}, \\
    \Dot{x} & = r x - h(y) - p(t) v(t), & x(t_{0}) = x_{0}, 
\end{align}

\begin{align*}
    & v(t) \in [-v_{1}, v_{2}], \\
    & a > 0.
\end{align*}

Функция Понтрягина и сопряженная система имеют вид. 
\begin{gather}
    {H}  = \psi_{1} \big( r x- h(y) - p(t) v \big) + \psi_{2} \big( v - a y \big),\\
    \Dot{\psi_{1}} = -H_x = -r \psi_{1},\label{p1}\\ 
    \Dot{\psi_{2}}  = -H_y = h'(y) \psi_{1} +  a \psi_{2}.\label{p2}
\end{gather}

Запишем  условия трансверсальности.
\begin{align}
    {\psi_{1}(T)} = \mathcal{- L}_x(T)= 1,\\
    {\psi_{2}(T)} = \mathcal {- L}_y(T)= p(T).
\end{align} 


Условия максимума функции Понтрягина:
\begin{align}
     \big (\psi_{2} - p(t)\psi_{1} \big )v \to \mathrm{max}. \label{p4}
\end{align}


${v^*}$ - типа "Bang-bang" определяется функцией переключения\\
$g = g(\psi_1 , \psi_2, p)= \big (\psi_{2} - p(t)\psi_{1} \big)$ и выглядит следующим образом:

\begin{gather}
v^*(\psi_1 , \psi_2, p) = 
 \begin{cases}
   v_{1}, &\text{ $\psi_{2} > p(t)\psi_{1}$},\\
   -v_{2}, &\text{$\psi_{2} < p(t)\psi_{1}$},\\
   [-v_{2},v_{1}], &\text{$\psi_{2} = p(t)\psi_{1}$}.
 \end{cases}\label{p3}
\end{gather}


Дадим следующую экономическую интерпретацию двойственным переменным и соотношениям принципа максимума:\\
 сопряженная переменная $ \psi_1(t) $ --- это теневая цена в момент времени ${t}$ одной денежной единицы, хранящейся на денежном счете,\\
 сопряженная переменная $ \psi_2(t) $ --- это теневая цена одного бушеля в момент времени ${t}$.
 
 
Уравнение (\ref{p1}) показывает, что при положительной ставке $ r>0 $ теневая цена денежных средств падает по экспоненте с коэффициентом $r$ в степени, т.е.  денежные средства обесцениваются с течением времени ввиду сокращения временного периода получения потенциального пассивного дохода.


Из уравнения (\ref{p2}) вытекает, что теневая цена пшеницы растет пропорционально предельным издержкам на хранение пшеницы (с учетом теневой цены денежных средств).


Из условия максимума (\ref{p3}) (см. также (\ref{p4})) вытекает, что  ${v^*}$ --- это скорость продажи, покупки или бездействия трейдера на бирже при условии оптимальной стратегии, продавать по максимально доступной  ставке стоимости бушеля, или покупать, если стоимость бушеля складывается выгоднее стоимости денежных средств.


В случае если теневая цена бушеля с учетом стоимости хранения точно равна биржевой стоимости бушеля, оптимальная стратегия трейдера на бирже не определяется из Принципа максимума.


\mysubsection{Вторая модификация задачи}


Добавим в исходную задачу коэффициент влияния условий хранения как функцию.\\
Введем новые обозначения:\\
${a(t)}$ = коэффициент влияния условий хранения на зерно.\\
$a(t) > 0$\\
 Наша задача запишется так:

\begin{align}
\mathcal{J} & = x(T) + p(T) y(T) \to \mathrm{max},\\
    \Dot{y} & = v - a(t) y, & y(t_{0}) = y_{0}, \\
    \Dot{x} & = r x - h(y) - p(t) v, & x(t_{0}) = x_{0}, 
\end{align}

\begin{align*}
    & v(t) \in [-v_{1}, v_{2}], \\
    & a(t) > 0.
\end{align*}

Функция Понтрягина и сопряженная система имеют вид. 
\begin{align}
    {H} & = \psi_{1} \big( r x- h(y) - p(t) v \big) + \psi_{2} \big( v - a(t)y \big),\\
    \Dot{\psi_{1}} & = -H_x = -r \psi_{1}, \label{p6}\\
    \Dot{\psi_{2}} & = -H_y = h'(y) \psi_{1} +  a(t) \psi_{2}.\label{p7}
\end{align} 

Запишем  условия трансверсальности.
\begin{align}
    {\psi_{1}(T)} = \mathcal{- L}_x(T)= 1,\\
    {\psi_{2}(T)} = \mathcal {- L}_y(T)= p(T).
\end{align} 


Условия максимума функции Понтрягина:
\begin{align}
     \big (\psi_{2} - p(t)\psi_{1}\big )v \to \mathrm{max}.
\end{align}


${v^*}$ - типа "Bang-bang" определяется функцией переключения\\
$g = g(\psi_1 , \psi_2, p)= \big (\psi_{2} - p(t)\psi_{1} \big)$ и выглядит следующим образом:

\begin{align}
v^*(\psi_1 , \psi_2, p) = 
 \begin{cases}
   v_{1}, &\text{ $\psi_{2} > p(t)\psi_{1}$},\\
   -v_{2}, &\text{$\psi_{2} < p(t)\psi_{1}$},\\
   [-v_{2},v_{1}], &\text{$\psi_{2} = p(t)\psi_{1}$}.
 \end{cases}\label{p8}
\end{align} 


Дадим следующую экономическую интерпретацию двойственным переменным и соотношениям принципа максимума:\\
 сопряженная переменная $ \psi_1(t) $ --- это теневая цена в момент времени ${t}$ одной денежной единицы, хранящейся на денежном счете,\\
 сопряженная переменная $ \psi_2(t) $ --- это теневая цена одного бушеля в момент времени ${t}$.
 
 
Уравнение (\ref{p6}) показывает, что при положительной ставке $ r>0 $ теневая цена денежных средств падает по экспоненте с коэффициентом $r$ в степени, т.е.  денежные средства обесцениваются с течением времени ввиду сокращения временного периода получения потенциального пассивного дохода.


Из уравнения (\ref{p7}) вытекает, что теневая цена пшеницы растет пропорционально предельным издержкам на хранение пшеницы (с учетом теневой цены денежных средств).


Из условия максимума (\ref{p8})  вытекает, что  ${v^*}$ --- это скорость продажи, покупки или бездействия трейдера на бирже при условии оптимальной стратегии, продавать по максимально доступной  ставке стоимости бушеля, или покупать, если стоимость бушеля складывается выгоднее стоимости денежных средств.


В случае если теневая цена бушеля с учетом стоимости хранения точно равна биржевой стоимости бушеля, оптимальная стратегия трейдера на бирже не определяется из Принципа максимума.
\mysubsection{Третья модификация задачи}

Добавим в исходную задачу коэффициент влияния условий хранения и противодействие плохим условиям хранения.\\

Введем новые обозначение:\\
${a(t)}$ = коэффициент влияния условий хранения на зерно.\\
$a(t) > 0$\\
${u(t)}$ = интенсивность поддержания необходимых условий хранения пшеницы в единицу времени.\\
$u(t) \in [0,1]. $\\

Наша задача запишется так:

\begin{align}
\mathcal{J} & = x(T) + p(T) y(T) \to \mathrm{max},\\
    \Dot{y} & = v(t) - a(t) \big (1 - u(t)\big ) y(t), & y(t_{0}) = y_{0},\\
    \Dot{x} & = r x - h(y) - p(t) v(t) - b u(t), & x(t_{0}) = x_{0}, 
\end{align}

В задаче два управления:

\begin{gather*}
    v(t) \in [-v_{1}, v_{2}], \\
     u(t) > 0, \,\;
    a(t) > 0, \,\;
     b > 0
\end{gather*}

Выпишем необходимые условия оптимальности для этой задача\\
\begin{align}
    {H} & = \psi_{1} \Big (r x- \frac{1}{2}|y| - p(t) v(t) - b u(t) \Big ) + \psi_{2} \Big (v(t) - a \big (1-u(t) \big  )y\Big).
\end{align} 

Запишем условия для сопряженных переменных и условия трансверсальности.
\begin{gather}
    \Dot{\psi_{1}} = -H_x = -r \psi,\,\;{\psi_{1}(T)} = 1, \label{g1}\\
    \Dot{\psi_{2}}  = -H_y = \psi_{1} h'(y) + a(1-u(t))\psi_{2},\;\, {\psi_{2}(T)} = p(T).\label{g2}
\end{gather} 

Условия максимума функции Понтрягина:\\
\begin{gather}
     \big (\psi_{2} - \psi_{1} p(t) \big ) v + (\psi_{2} a(t) y - b \psi_{1})u(t)  \to \mathrm{max},\\
      u(t) > 0,\nonumber\\ 
     v(t) \in [-v_{1}, v_{2}]. \nonumber
\end{gather}


${v^*}$ и ${u^*}$- типа "Bang-bang" определяется функциями переключения\\
$g = g(\psi_1 , \psi_2, p)= \big (\psi_{2} - p(t)\psi_{1} \big)$ и $g_1 = g_1(\psi_1 , \psi_2, p)= \big (\psi_{2} a(t) y - b \psi_{1} \big)$ и выглядят следующим образом:

\begin{align}
v^*(\psi_1 , \psi_2, p) = 
 \begin{cases}
   v_{1}, &\text{ $\psi_{2} > p(t)\psi_{1}$},\\
   -v_{2}, &\text{$\psi_{2} < p(t)\psi_{1}$},\\
   [-v_{2},v_{1}], &\text{$\psi_{2} = p(t)\psi_{1}$};
 \end{cases}\label{g3}
\end{align} 

\begin{align}
u^*(\psi_1 , \psi_2, p, a(t),b) = 
 \begin{cases}
   u_{1}, &\text{ $\psi_{2} a(t) y > b \psi_{1}$},\\
   -u_{2}, &\text{$\psi_{2} a(t) y < b \psi_{1}$},\\
   [-u_{2},u_{1}], &\text{$\psi_{2} a(t) y = b \psi_{1}$}.
 \end{cases}\label{g4}
\end{align} 

Дадим следующую экономическую интерпретацию двойственным переменным и соотношениям принципа максимума:\\
 сопряженная переменная $ \psi_1(t) $ --- это теневая цена в момент времени ${t}$ одной денежной единицы, хранящейся на денежном счете,\\
 сопряженная переменная $ \psi_2(t) $ --- это теневая цена одного бушеля в момент времени ${t}$.
 
 
Уравнение (\ref{g1}) показывает, что при положительной ставке $ r>0 $ теневая цена денежных средств падает по экспоненте с коэффициентом $r$ в степени, т.е.  денежные средства обесцениваются с течением времени ввиду сокращения временного периода получения потенциального пассивного дохода.


??Из уравнения (\ref{g2}) вытекает, что теневая цена пшеницы растет пропорционально предельным издержкам на хранение пшеницы (с учетом теневой цены денежных средств).


??Из условия максимума (\ref{g3}), (\ref{g4})  вытекает, что  ${v^*}$ --- это скорость продажи, покупки или бездействия трейдера на бирже при условии оптимальной стратегии
%, продавать по максимально доступной  ставке стоимости бушеля, или покупать, если стоимость бушеля складывается выгоднее стоимости денежных средств.
; ${u^*}$ --- это интенсивность поддержания необходимых условий хранения пшеницы. 


??В случае если теневая цена бушеля с учетом стоимости хранения точно равна биржевой стоимости бушеля, оптимальная стратегия трейдера на бирже не определяется из Принципа максимума.

