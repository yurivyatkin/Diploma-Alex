Обозначения:
 
{Т} = горизонт времени,\\
{х(t)} = остаток денежных средств в долларах в момент времени t,\\
{у(t)} = баланс пшеницы в бушелях в момент времени t,\\ 
{u(t)} = скорость покупки пшеницы в бушелях за единицу времени; отрицательная покупка означает продажу,\\
{р(t)} = цена пшеницы в долларах за бушель в момент времени t,\\
{r} = постоянная положительная процентная ставка, заработанная на денежном балансе,\\
{h(у)} = стоимость за время хранения пшеницы.

\section{Первая модификация задачи}

Добавим в исходную задачу коэффициент влияния условий хранения как постоянное число.\\

Введем новые обозначение:\\
{a} = коэффициент влияния условий хранения на зерно.\\
a - const\\
Наша задача запишется так:

\begin{align}
    \Dot{y} & = v(t) - a, & y(t_{0}) = y_{0} \\
    \Dot{x} & = r x - h(y) - p(t) v(t), & x(t_{0}) = x_{0} \\
    \mathcal{L} & = x(T) + p(T) y(T) \to \mathrm{max}
\end{align}

\begin{align}
    & v(t) \in [-v_{1}, v_{2}], \\
    & a > 0
\end{align}




\section{Вторая модификация задачи}


Добавим в исходную задачу коэффициент влияния условий хранения как функцию.\\
Введем новые обозначения:\\
{a(t)} = коэффициент влияния условий хранения на зерно.\\
a(t) = [0, 1]\\
 Наша задача запишется так:\\

\begin{align}
    \Dot{y} & = v(t) - (1 - a(t)) y(t), & y(t_{0}) = y_{0} \\
    \Dot{x} & = r x - h(y) - p(t) v(t), & x(t_{0}) = x_{0} \\
    \mathcal{J} & = x(T) + p(T) y(T) \to \mathrm{max}
\end{align}

\begin{align}
    & v(t) \in [-v_{1}, v_{2}], \\
    & a(t) = [0, 1]\\
    & a > 0
\end{align}

Функция Понтрягина и сопряженная система имеют вид. 
\begin{align}
    \mathcal{H} & = \psi_{1} (r x- h(y) - p(t) v(t)) + \psi_{2} (v(t) - (1-a(t))y(t))\\
    \Dot{\psi_{1}} & = -H x = -r \psi_{1},\\
    \Dot{\psi_{2}} & = -H y = h'(y) \psi_{1} + (1- a(t))\psi_{2}
\end{align} 

Запишем  условия трансверсальности.
\begin{align}
    {\psi_{1}(T)} = - {L}x(T)= 1,\\
    {\psi_{2}(T)} = - {L}y(T)= p(T)
\end{align} 


Условия максимума функции Понтрягина:\\
\begin{align}
     (\psi_{2} - p(t)) \to \mathrm{max}\\
\end{align}

\section{Третья модификация задачи}

Добавим в исходную задачу коэффициент влияния условий хранения и противодействие плохим условиям хранения.\\

Введем новые обозначение:\\
{a(t)} = коэффициент влияния условий хранения на зерно.\\
a(t) = [0, 1]\\
{u(t)} = интенсивность поддержания необходимых условий хранения пшеницы в единицу времени.\\
u(t) = [0, 1]\\

Наша задача запишется так:

\begin{align}
    \Dot{y} & = v(t) - a(t)(1 - u(t)) y(t), & y(t_{0}) = y_{0} \\
    \Dot{x} & = r x - h(y) - p(t) v(t) - b u(t), & x(t_{0}) = x_{0} \\
    \mathcal{J} & = x(T) + p(T) y(T) \to \mathrm{max}
\end{align}

В задаче два управления:

\begin{align}
    & v(t) \in [-v_{1}, v_{2}], \\
    & u(t) \in [0, 1], \\
    & a(t) \in [0, 1],\\
    & b > 0
\end{align}

Выпишем необходимые условия оптимальности для этой задача\\
\begin{align}
    \mathcal{H} & = \psi_{1} (r x- \frac{1}{2}|y| - p(t) v(t) - b u(t)) + \psi_{2} (v(t) - a(1-u(t))y)
\end{align} 

Запишем условия для сопряженных переменных и условия трансверсальности.
\begin{align}
    \Dot{\psi_{1}} & = -H x = -r \psi,\\
    \Dot{\psi_{2}} & = -H y = \psi_{1} h'(y) + a(1-u(t))\psi_{2},\\
    {\psi_{1}(T)} = 1,\\
    {\psi_{2}(T)} = p(T),
\end{align} 

Условия максимума функции Понтрягина:\\
\begin{align}
     (\psi_{2} - \psi_{1} p(t)) v(t) + (\psi_{2} a(t) y - b \psi_{1})u(t)  \to \mathrm{max}\\
     & u(t) \in [0, 1],\\
     & v(t) \in [-v_{1}, v_{2}]
\end{align}



