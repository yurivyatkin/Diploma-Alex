В выпускной квалификационной работе рассмотрена возможность использования теории оптимального управления при решении задач связанных с выбором наилучшей стратегии трейдинга, на примере торговли зерном на бирже.


В работе рассмотрен базовый вариант решения этой задачи \cite{b8}, взятый за основу. 

Задача состоит в том, чтобы максимизировать целевой функционал

\begin{align}
    J = x(T) + p(T) y(T)\to \mathrm{max} \nonumber
\end{align} 
с ограничениями:
\begin{align}
    \Dot{x} & = r x(t) - h(y) - p(t) v(t),\, x(0) = x_{0}, \nonumber\\
    \Dot{y} & = v(t), \, y(0) = y_{0}, \nonumber \\
    - v_2 & \le v(t) \le v_1. \nonumber
\end{align}    


Данная базовая модель описывает функцию оптимального управления для различных параметров ($x(t)$, $y(t)$,${v(t)}$, $p(t)$, ${r}$, $h(y)$) стратегии покупки и продажи пшеницы, с целью максимизации общей стоимости активов трейдера в конечный момент рассматриваемого промежутка времени. Проведено аналитическое исследование модели с помощью Принципа максимума Понтрягина. 


В результате анализа на предложенном горизонте времени определенны временные периоды для оптимально выгодной покупки и продажи зерна для достижения основной цели – максимизации общей стоимости актива.


Установлено, что оптимальная стратегия позволяет увеличить общую стоимость активов в конечный момент времени по сравнению со стратегией бездействия трейдера. Причина этого состоит в том, что прогнозируемый скачок цены зерна дает возможность получить дополнительную выгоду при заданном уровне затрат на хранение пшеницы. Модификация базовой задачи произведена включением квадратичной функцией h описывающей затраты на хранение зерна.


Предложены три модификации исходной модели, учитывающие условия хранения зерна.


В первой модификации предполагается, что зерно портится (из-за плохих условий хранений) с постоянной скоростью. Во второй модификации скорость порчи зерна является функцией времени и зависит, например, от погодных условий. Третья модификация содержит новое управление, описывающее интенсивность усилий трейдера по повышению условий хранения зерна.
