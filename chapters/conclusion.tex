В выпускной квалификационной работе рассмотрена возможность использования теории оптимального управления при решении задач связанных с выбором наилучшей стратегии трейдинга, на примере торговли зерном на бирже.


В работе рассмотрен базовый вариант решения этой задачи \cite{b8}, взятый за основу. 

Задача состоит в том, чтобы максимизировать целевой функционал

\begin{align}
    J = x(T) + p(T) y(T)\to \mathrm{max} \nonumber
\end{align} 
при ограничениях:
\begin{align}
    \Dot{x} & = r x(t) - h(y) - p(t) v(t),\,\; x(0) = x_{0}, \nonumber\\
    \Dot{y} & = v(t), \,\; y(0) = y_{0}, \nonumber \\
    - v_2 & \le v(t) \le v_1. \nonumber
\end{align}    

Задача заключается в отыскании стратегии покупки и продажи пшеницы,  максимизирющей общую стоимость активов трейдера в конечный момент рассматриваемого промежутка времени. Проведено аналитическое исследование модели с помощью Принципа максимума Понтрягина (в том числе, при заданных значениях параметров).


В результате анализа на предложенном горизонте времени определенны временные периоды для оптимально выгодной покупки и продажи зерна для достижения основной цели – максимизации общей стоимости активов.


Установлено, что оптимальная стратегия позволяет увеличить общую стоимость активов в конечный момент времени по сравнению со стратегией бездействия трейдера. Причина этого состоит в том, что прогнозируемый скачок цены зерна дает возможность получить дополнительную выгоду при заданном уровне затрат на хранение пшеницы. Модификация базовой задачи произведена рассмотрением  квадратичной функцией $h$ описывающей затраты на хранение зерна.


Необходимо отметить, что при применении данной модификации функция оптимального управления не изменилась (относительно исходной постановки задачи).

Наконец, предложены три модификации исходной модели, учитывающие условия хранения зерна.


В первой модификации предполагается, что зерно портится (из-за плохих условий хранения ) с постоянной скоростью. Во второй модификации скорость порчи зерна является функцией времени и зависит, например, от погодных условий. Третья модификация содержит новое управление, описывающее интенсивность усилий трейдера по повышению условий хранения зерна.


Для каждой модели конкретизированы соотношения Принципа максимума Понтрягина.


??Таким образом, все постановленные в выпускной квалификационной работе задачи выполнены, цель достигнута.

??Таким образом, выпускная квалификационная  работа демонстрирует возможности применения теории оптимального управления трейдингам на примере биржевой торговли.
