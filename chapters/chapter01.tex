\mysubsection{Математическое моделирование }

Математика в современном мире нашла широкое применение во всех отраслях. Для различных областей науки и в повседневной жизни математика выступает как инструмент широкой цифравизации, количественных расчетов, с использованием математических средств формулируются цели, задачи в процессе научных исследований, создаются модели производственных процессов и т.д.


Математическое моделирование завоевало основные позиции в исследованиях, оценке и получении будущих технологических решений. Математическое моделирование и цифравизация в настоящее время ставит перед собой глобальные задачи управления и регулирования, охватывая все области человеческой деятельности, природы и социума.
Математическая модель представляет собой совокупность математических объектов и отношений между ними, адекватно отражающих свойства и поведение исследуемого объекта.


Математическая модель отображает определенным образом, необходимые и достаточные признаки-параметры для описания изучаемого объекта или явления (технологического, физического, экономического и т.д.). При этом необходимым условием является установление определённых соответствий между характерными параметрами исследуемых объектов, явлений с математическими понятиями. 


Применяя математическую формализацию выбранные характеристики, отношения, связи внутри исследуемого объекта или явления приводятся в необходимое соответствие с математическими понятиями. Таким образом, используя математические уравнения, равенства, соотношения, формулы могут быть отображены существующие, вычисленные или предполагаемые параметры исследуемого объекта. Полученная совокупность математических формул, уравнений, равенств, соотношений является математическим описанием исследуемого объекта -  математической моделью исследуемого объекта. 


 Математическое моделирование имеет следующие преимущества: 
 
 
Экономическая эффективность (в частности, экономия ресурсов реальной системы);

Возможность моделирования гипотетического, то есть неосуществимого в природе объекта (прежде всего на разных этапах проектирования);


Возможность реализации опасных или трудно воспроизводимых режимов в природе (критический режим ядерного реактора, работа системы ПРО);


Возможность изменения шкалы времени; 

Простота многих аспектов анализа;


Большая предсказательная сила, благодаря возможности определения общих закономерностей;


Универсальность выполняемой технической и программной работы (компьютер, системы программирования и пакеты прикладных программ многоцелевого назначения).\cite{b6}



При построении математической модели изучаемого объекта или явления, производят определения тех  признаков и деталей, которые, с одной стороны, содержат более или менее полную информацию об объекте, а с другой стороны, допускают математическую формализацию. Математическая формализация означает, что соответствующие и адекватные математические понятия могут быть приведены в соответствие с особенностями и деталями объекта: числами, функциями, матрицами и так далее. Тогда отношения и связи, найденные и предполагаемые в исследуемом объекте, между его отдельными деталями и составными частями могут быть записаны с использованием математических отношений: равенств, неравенств, уравнений. Результатом является математическое описание изучаемого процесса или явления, то есть его математическая модель.



Основные этапы математического моделирования


Математическое моделирование представляет собой сложный процесс, состоящий из ряда последовательных этапов.  Во многом процесс построения математической модели опирается на профессиональные качества разработчика, накопленные базы эмпирических данных.


Выделяют следующие этапы: 


1.	Постановка цели, сбор данных об объекте исследования



Первый этап включает определение цели и задачи исследования, производится выбор объекта исследования. Устанавливаются основные параметры, характеристики исследуемого объекта, механизмы, взаимосвязи, различные факторы, оказывающие влияния на состояние исследуемого объекта. Производится анализ теоретической информации и ранее накопленных данных по существующим моделям исследуемого объекта. Определяются входные и выходные данные, упрощающие предположения принимаются. На данном этапе определяется первоначальное, условное описание объекта исследования, где отражены существенные количественные характеристики и основные параметры объекта, с учетом обоснованных упрощений. При этом необходимо соблюдать условия полноты и точности учета основных свойств и качеств объекта исследования.

2.Формализация.


Второй этап, как правило, содержит расчетную схему и математическое описание расчётной схемы. Отношения между параметрами исследуемого объекта записываются в виде математических выражений. Такое формальное описание называется математической моделью. На данном этапе могут еще отсутствовать значения некоторых параметров. 
Объект исследования может иметь несколько математических моделей. Это связано, прежде всего, с тем, что рассмотренные конструктивные схемы различаются по степени упрощения. Однако даже при рассмотрении одной и той же расчётной схемы можно построить принципиально разные математические модели. \cite{b3}


Так же разработанные ранее математические модели могут соответствовать уже существующим расчётным схемам, что во многом упрощает работу.


3.	Выбор рабочей математической модели.


Третий этап, для контроля корректности полученной системы математического моделирования, включает ряд проверок: оценочный количественный и качественный анализ построенной математической модели. На данном этапе могут быть выявлены некоторые противоречия, так же могут возникнут обоснованные основания для исключения при дальнейшем рассмотрении ряда количественных и качественных параметров объекта исследования. Результатом анализа на данном этапе является выбор рабочей математической модели.


4. Выбор метода решения.


Четвертый этап содержит построение или выбор численного метода. На выбор метода существенно влияют предпочтения разработчика, предпочтения и знания пользователей, доступность ресурсов, требования к решению и др. Корректное решение одной вычислительной задачи обычно может быть выполнено несколькими методами.


5. Реализация модели.


На пятом этапе разрабатывается эффективный вычислительный алгоритм, с помощью которого можно получить результат с необходимой точностью и в доступное время. К вычислительным алгоритмам предъявляют ряд требований, таких как экономичность, простота, корректность, точность и др. \cite{b4}


В дальнейшем разрабатывается программа реализации вычислительного алгоритма. После устранения всех выявленных недостатков приступают к проведению вычислительного эксперимента.


6. Анализ полученной информации.


Полученное и ожидаемое решения сравниваются, проводится контроль ошибок моделирования.


7. Проверка адекватности реального объекта.


Результаты, полученные моделью, сравниваются либо с имеющейся информацией об объекте, либо проводится эксперимент, а его результаты сравниваются с расчетными.\cite{b3}

\mysubsection{Основные принципы построения математических моделей }

В настоящий момент определен ряд свойств математической модели. Рассмотрим основные свойства математической модели: адекватность, точность, полнота, свойство производительности, экономичность: \cite{b1}, \cite{b3}
 
1.	Адекватность. Данное  свойство является одним из важнейших требований, предъявляемых к математической модели. Под этим свойством понимают, достаточно высокой точности, количественное  и качественное описание параметров исследуемого объекта или явления. Следует рассматривать адекватность математической модели именно по тем характеристикам, которые определены как существенные для данного исследуемого объекта. 


2.	Точность математической модели. Максимальное повышение точности математической модели, может привести к ее необоснованному усложнению и как следствие снижению ее экономичности.


3.	Полнота. Полнота математической модели должна обеспечить учет всех существенных для данного объекта исследований свойств и качеств. 


4. Производительность. Наличие возможности получение достоверных исходных данных  определяет продуктивность математических моделей. Математическая модель считается непродуктивной, в том случае, если получение таких данных по какой либо причине невозможно. В этом случае использование математической модели не целесообразно.

5.Экономичность. Вся совокупность затрат – материальных, человеческих, временных и др. определяет степень экономичности математических моделей.


Для создания математической модели, отвечающей всем вышеперечисленным требованиям, были разработаны основные принципы построения математической модели.


1.	Построение математической модели с довольно ограниченным диапазоном адекватности. 


На начальном этапе математического моделирования, как правило, необходимый диапазон изменения параметров объекта исследования не известен. С расширением диапазона адекватности, происходит усложнение математической модели. Построение математической модели с более широкой областью адекватности потребует дополнительных затрат, поэтому необходимо одновременно учитывать цель моделирования и экономическую обоснованность усложнения модели. Таким образом целесообразней начинать с построения математической модели с ограниченным диапазоном адекватности, постепенно усложняя модель.


2. Принцип постепенного усложнения математической модели.


Принцип постепенного усложнения математической модели заключается в том, что  построение математической модели необходимо начинать с разработки наиболее грубой-простой модели, опираясь при этом на самые существенные характеристики исследуемого объекта. Контролируя пригодность математической модели на каждом этапе моделирования и, в случаи необходимости, производя, с учетом полученных данных, корректировку модели, происходит постепенное усложнение математической модели. Данный цикл может повторятся до тех пор пока не будет достигнута цель моделирования и не будет получена подходящая математическая модель.


3. Принцип согласованности.


Погрешность исходных данных и точность математической модели должны быть согласованны. С уменьшением значений ошибки исходных данных, возрастает точность математической модели, и соответственно, при увеличении ошибки в исходных данных, точность математической модели должна уменьшаться. Если погрешность исходных данных выше значения количественной характеристики точности модели, то модель не будет соответствовать требованию производительности и ее использование теряет всякий смысл. Если ошибка исходных данных намного меньше значения количественной характеристики точности модели, то модель может быть недостаточно точной и не соответствовать требованию полноты.\cite{b1}, \cite{b3}


4. Принцип перехода к стохастической математической модели.


При построении математической модели возникают ситуации, когда достаточно сложно или практически невозможно определить ряд параметров, зависимостей с достаточной точностью. В таких ситуациях прибегают к рассмотрению случайных функций или величин, то есть к рассмотрению стохастических математических моделей. Основная трудность анализа стохастической математической модели обычно связана с тем, что необходимые характеристики случайных величин или случайных функций часто не известны или известны с низкой точностью.


\mysubsection{ Классификация математических моделей} 
Рост количества математических моделей различных видов, обусловлен расширением областей их применения (планирование, прогнозирование, управление технологическими, социальными, экономическими процессами), а так же развитием методов моделирования. Поэтому в настоящее время нет единой системы классификации математических моделей.


Математические модели классифицируют по различным категориям или классификационным признакам например: по области применения, по способу описания, по фактору времени и др.


Приведем пример классификации математических моделей по некоторым категориям.


1.	По фактору времени различают статические и динамические математические модели. 


Статические математические модели описывают установившиеся процессы, не зависящие от времени.


Динамические математические модели описывают неустановившиеся процессы, где с течением времени происходят видимые изменения.


2.	По характеру  процесса моделирования различают детерминированные и стохастические математические модели.


В детерминированных математических моделях каждой характеристике, каждому параметру соответствует определенное значение, либо соответствующая функция. Детерминированная модель строится в тех случаях, когда факторы, влияющие на результат операции, могут быть измерены или оценены достаточно точно, а случайные факторы либо отсутствуют, либо ими можно пренебречь.


Стохастические математические модели применяются в тех случаях , когда все параметры модели или какая то их часть представляют собой случайные величины. 


3.	В соответствии с целью моделирования различают описательные математические модели, оптимизационные математические модели и управленческие математические модели.


Описательные - дескрипторные математические модели используются для установления закономерностей изменения параметров исследуемых объектов и явлений. 


Оптимизационные модели используются для определения наилучших параметров исследуемого объекта, на основе определенного критерия, или для определения   оптимального способа управления. 


При построении оптимизационных математических моделей применяются одна или несколько описательных моделей, и используется определенный критерий для определения оптимальных выходных значений. 


Управленческие модели используются в тех ситуациях, когда необходимо определить наиболее эффективное управленческое решение. При этом обычно выбор наиболее эффективных решений производится из заданного набора альтернатив, а общий процесс принятия решений представляет собой последовательность таких выборов.


4. По способу получения различают теоретические математические модели и экспериментальные математические модели.


Теоретические модели формируются на основе описания физических процессов функционирования объекта.


Экспериментальные модели формируются на основе поведения объекта во внешней среде, рассматривая его как «черный ящик». Эксперименты при этом могут быть физические (на техническом объекте или на его физической модели) или вычислительные (теоретической математической модели). 


\mysubsection{Оптимальное управление } 
В двадцатом веке, с развитием масштабных производств и ограничением ресурсов, критически встала задача оптимального использования энергии, материальных ресурсов, рабочего времени, вопросы наилучшего управления различными процессами в физике, технологии, экономика, экологии и т.д. стали более актуальными.


Бурное развитие теории оптимального управления связано с повышением требований как к скорости и точности регулирующих систем, так и к переходу к рыночной экономике. Увеличение скорости возможно только при правильном распределении ограниченных ресурсов управляющих систем, и поэтому основной задачей теории оптимального управления стала задача ограничения управляющих воздействий. При построение высокоточных систем управления возникла необходимость учитывать взаимное влияние отдельных частей системы. Синтез таких сложных многомерных систем также является предметом теории оптимального управления.


Разнообразными задачами оптимального управления являются выбор наиболее выгодных режимов управления сложными динамическими объектами: в механике полета; в технике; в энергии; в экономике; на бирже (поиск стратегии для оптимальных инвестиций) и т.д.


Теория оптимального управления включает в себя элементы теории управления движением, функционального анализа, исследования операций, математического программирования и теории игр. В широком смысле решение проблемы оптимального управления означает разработку наилучшего закона управления или набора контрольных действий для данного объекта или процесса. Для этого строится математическая модель объекта или процесса, которая описывает его поведение во времени под влиянием управляющих воздействий и текущего состояния. Включает в себя цель управления, выраженную через критерий качества; динамика объекта в виде дифференциального, интегрального, конечно-разностного или другого уравнения, описывающего метод движения объекта управления; ограничения на ресурсы, используемые в виде уравнений или неравенств.


Основные принципы общей теории оптимального управления были заложены в 1956–1961 гг. В школах Л.С. Понтрягина и Р. Беллмана, хотя, конечно, на практике проблемы оптимального управления встречались и раньше. Центральный результат математической теории оптимального управления - принцип максимума Понтрягина, являющийся необходимым условием оптимальности в задаче оптимального управления, был выдвинут автором в качестве гипотезы в 1955 г., а затем доказан его учениками (Р.В. Гамкрелидзе для линейного случая, В.Г. Болтянский для общего случай нелинейной задачи с функциональными ограничениями). Принцип максимума Понтрягина послужил мощным стимулом к пересмотру основных понятий теории экстремума, ее развитию и лёг в основу огромного количества исследований и новых результатов.



Ценность математической теории процессов оптимального управления заключается в том, что она обеспечивает единую методологию для решения очень широкого круга задач оптимального проектирования и управления, устраняет инерцию и отсутствие общности прежних конкретных методов и способствует получению значимых результатов и методов  в смежных областях. 
